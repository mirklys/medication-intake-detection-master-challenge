
%  overview of grading criteria:
%  -----------------------------
%  Structure and readability
%  The structure is clear and easy to follow. Language is clear and precise. 
%  All the key concepts are well introduced. There is a good balance between 
%  comprehension and concision.

%  Executive summary
%  Is informative and covers all the key points of the report thoroughly. It 
%  is standalone and provides a clear picture of the report. Is concise and 
%  to the point. Is engaging and strongly motivates the reader to read the whole 
%  report.

%  Introduction
%  Clear and complete identification of project goals and objectives. The project 
%  and its objectives are well motivated and the background is described clearly. 
%  The length and level of required details are very well balanced.

%  Body
%  Content is comprehensive, accurate, and persuasive. Major points are stated 
%  clearly and are well supported.

%  Evaluation
%  An extensive and clear evaluation of the outcomes of the project with respect 
%  to the project goals and objectives and also compared to previous existing 
%  solutions/systems. The strengths and weaknesses of the outcome and progress 
%  are critically and deeply analyzed

%  Conclusion and suggestions
%  Conclusions are brief, clean-cut and specific, and relate specifically to the 
%  objectives of the project as set out in the introduction. The conclusions follow 
%  logically from the outcomes of the projects. Suggestions are logically connected 
%  to the conclusions, they are action-oriented, feasible, and presented in order of 
%  importance.
 

\documentclass[a4paper,12pt]{article}

% Packages
\usepackage[utf8]{inputenc}
\usepackage{graphicx}    % For including images
\usepackage{amsmath}     % For math
\usepackage{amsfonts}    % For fonts
\usepackage{hyperref}    % For hyperlinks
\usepackage{geometry}    % Page margins
\usepackage{caption}     % Custom captions
\usepackage{float}       % Precise figure placement
\usepackage{natbib}      % Bibliography
\geometry{margin=1in}

% Title and Author
\title{MedAssist: An Automated Solution for The Assessment of Medication Intake}
\author{Adem Kaya, George Lalidis, Giedrius Mirklys, Tam Van}
\date{21-06-2025}
\begin{document}

% Title Page
\maketitle

\section{Introduction}
For the course AI in the Professional Workfield (SOW-MKI76), we were challenged 
to develop a project for a company in a selection of companies provided on the 
Masters Challenge platform. Given our shared background and passion for societal 
impact and healthcare, we ended up with a company called MedAssist.

\subsection{MedAssist}
MedAssist is a company that is concerned with building medication dispensary devices. 
Not only do their devices feature automatic release of medication, such that it 
capable of helping its patients to take their medication on time, but the devices
also feature a build in camera that is capable of recoding videos of the patients 
whenever they are exactly in front of the device. 

\subsection{The Problem}
As of now the video material that is collected by the deviced is manually reviewed by
humans to check whether the patient in question has sucessfully taken their medication. 
The problem lays in the time consuming nature of this process. In order to provide 
a solution to this time consuming approach, MedAssist has reached out to us to 
build an AI which is capable of automatically assessing whether the patient has taken
their medication or not.

\subsection{The Goal}
In order to provide a solution to the problem, the goal is to build an AI which 
is capable of automatically assessing whether the patient has taken their medication
or not to its best extent. Not only should we aim to maximize the accuracy of the
AI, but we should also carefully aim to minimize the amount of false positives as
we do not want the AI to make it seem like the patient has taken their medication 
even though they have not.

\section{Methods}

\subsection{Provided Data}
\subsection{Human Activity Detection (HAD)}
\subsection{Finetuning}


\section{Results}
Here is how you include an image (make sure the image is in the same directory or provide the path):

\begin{figure}[H]
    \centering
    \includegraphics[width=0.5\textwidth]{example-image} % Replace with your image file
    \caption{Sample Image}
    \label{fig:sample-image}
\end{figure}

\section{Tables}
You can include tables like this:

\begin{table}[H]
    \centering
    \begin{tabular}{|c|c|c|}
        \hline
        Column 1 & Column 2 & Column 3 \\
        \hline
        A & B & C \\
        1 & 2 & 3 \\
        \hline
    \end{tabular}
    \caption{Sample Table}
    \label{tab:sample-table}
\end{table}

\section{Conclusion}
Summarize the results and discuss the implications of your findings.

\section{Discussion and Recommendations}

\section*{References}

\end{document}
